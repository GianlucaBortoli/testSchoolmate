\documentclass{sig-alternate-05-2015}
\usepackage{subscript}
\usepackage{tikz}
\usepackage{hyperref}
\usepackage{graphicx}
\usepackage{amsmath}
\usepackage{listings}
\usepackage[lined,boxed,longend,ruled]{algorithm2e}
\newcommand\mycommfont[1]{\footnotesize\ttfamily\textcolor{blue}{#1}}
\SetCommentSty{mycommfont}
\SetKwProg{Fn}{def}{\string:}{end}
\SetKwFunction{CompGrid}{compute\_grid}
\SetKwFunction{CompTop}{compute\_topics}
\SetKwFunction{CompComArea}{compute\_intersection}
\SetKwFunction{CompDiffArea}{compute\_difference}
\SetKwFunction{CrDT}{create\_dataframes}
\SetKwFunction{SeqCom}{sequential\_computation}
\SetStartEndCondition{ }{}{}%
\SetKw{KwTo}{in}\SetKwFor{For}{for}{\string:}{end for}%
\newcommand{\forcond}{$df$ \KwTo{$dist\_clust\_pages$}}
\newcommand{\forRecDiff}{$map$ \KwTo{$commonAreas$}}
\newcommand{\forRecCom}{$map$ \KwTo{$computedMaps$}}
\newcommand{\forNaive}{$page$ \KwTo{$pages$}}
\newcommand{\forNaiveCom}{$cell$ \KwTo{$squares$}}
\newcommand\lword[1]{\leavevmode\nobreak\hskip0pt plus\linewidth\penalty50\hskip0pt plus-\linewidth\nobreak#1}
\AlgoDontDisplayBlockMarkers\SetAlgoNoEnd\SetAlgoNoLine%

% listings PHP code color
\definecolor{dkgreen}{rgb}{0,.6,0}
\definecolor{dkblue}{rgb}{0,0,.6}
\definecolor{dkyellow}{cmyk}{0,0,.8,.3}
\lstset{
  language        = php,
  basicstyle      = \small\ttfamily,
  keywordstyle    = \color{dkblue},
  stringstyle     = \color{red},
  identifierstyle = \color{dkgreen},
  commentstyle    = \color{gray},
  emph            =[1]{php},
  emphstyle       =[1]\color{black},
  emph            =[2]{if,and,or,else},
  emphstyle       =[2]\color{dkyellow}
}


\begin{document}

\title{Security Testing Report of Schoolmate}

\numberofauthors{1}
\author{
    Delivery date: \today\\
    Security reviewer: Gianluca Bortoli\\
    Student ID: 179816\\
    Email: gianluca.bortoli@studenti.unitn.it\\
    Beneficiary: Ceccato Mariano
}
\maketitle


\section{Introduction}
This work aims at performing a security analysis study on \emph{Schoolmate}.
This web application is a PHP/MySQL solution for elementary,
middle and high schools where different type of users can manage all the needed
information to fulfill their everyday tasks.

This report is structured as follows. Section~\ref{tests} describes the naming
convention used for the test cases, how they are structured into packages and shows
the vulnerabilities that have been found on the subject application.
Section~\ref{fixes} depicts how to fix these security braches. Section~\ref{outcomes}
reports the outcomes of the Proof-of-Concept (PoC) attacks described in Section~\ref{tests}.
Finally, Section~\ref{steps} depicts the steps that have to be taken in order to
be able to run all the tests from scratch.


\section{Security test cases}\label{tests}
This security review uses a static analysis tool to scan the application's source code
in order to spot security vulnerabilities.
\emph{Pixy} is able to scan PHP applications to find security problems at the source code level.
This software is used in order to spot all the possible vulnerabilities in \emph{Schoolmate},
the subject web application.

The workflow followed to develop this work is the following:
\begin{enumerate}
    \item \label{pixy} Pixy is run on the application source code to produce vulnerability reports
        regarding Cross Site Scripting attacks (XSS).
    \item \label{classification} all the reports are manually analyzed to perform a first
        classification of the resulting vulnerabilities. Thus, each of them has to be
        categorized either as \textbf{false positive}, \textbf{reflected XSS} or
        \textbf{stored XSS} (please refer to the \lword{\emph{xss\_classification.pdf}}
        attachment for a full description of all the vulnerabilities found in step \ref{pixy}).
    \item a test cases for all the true positives (TP)\footnote{A single test can cover
        either a reflected or a stored XSS vulnerability. Furthermore, each Pixy report may
        include more than one vulnerability for each report.} is written in the form
        of a PoC attack against the application under test.
    \item every test case is checked to be failing on the bugged application to demonstrate
        \emph{Schoolmate} has such security vulnerabilities and that they can be exploited.
    \item \label{fix} the \emph{Schoolmate}'s PHP code is modified to fix the vulnerabilities
        found in step \ref{classification}.
    \item the whole test suite is run again to check all the tests pass. This means that
        the PoC attack implemented in each test case is no longer possible after the application
        is fixed.
    \item Pixy is run again on the new application source code to check the
        vulnerabilities are really fixed and to make sure step \ref{fix} did not
        introduce any additional security breach.
\end{enumerate}

The most effective way to implement a security test case for a XSS vulnerability is to
be able to inject some malicious code inside the HTML of the page that is shown to the user.
All the tests implemented in this work inject a \emph{malicious link} and then checks
whether it is found on the target web page.

\subsection{Test case code structure}
The test suite developed to demonstrate that the application under revision has
security concerns makes use of JWebUnit, a Java-based testing framework for web
applications.

The test suite's source code is structured as follows:
\begin{itemize}
    \item the \textbf{src/main/java} folder contains a Java file for each Pixy report which
        has at least one TP. The test file naming follows the convention
        \lword{\emph{Test<pixy\_report\_number>.java}} so that it is possible to easily match the test
        case with its Pixy report counterpart\footnote{This is done to simplify the revision work,
        since Pixy identifies every report with a number in its filename.}.
    \item the \textbf{src/main/java/common} folder contains three classes that implement
        utility functions that are available to all the test cases. These functions
        allows not to soil the actual test case implementation, separating all the
        sequences of operations that has to be repeated in many tests in order to navigate
        the web page depending on the scenario. In this way, it is possible to maximize
        the code reuse across multiple test cases and to better modularize the project.
\end{itemize}
A noteworthy aspect is how tests' assertions are written. When dealing with
security PoC test cases, the security expert writes a test that is thought to fail
if the vulnerability can be exploited and it should pass otherwise. Hence, in this scenario
a failing test (and its assertion) means that the vulnerability is still present,
while a passing one means that the vulnerability is not present any more.

For the seek of generality, all the test cases can run in any order. To achieve this,
every test cases involving a stored XSS vulnerability has a $cleanup$ function called
after the test case execution which removes potentially dirty data from the database.

Furthermore, most of the tests perform three types of action on the web page.
They can \emph{navigate} through the pages of the web application clicking buttons and/or links,
they can \emph{input} some values in the available forms and text fields and finally they
\emph{make assertions} about the content of the page (eg. that the malicious link is not
present in the page).

Last but not least, some of the test cases (ie. the ones involving the variable $page2$)
required to modify the structure of the Document Object Model (DOM) of the webpage, since
the submit buttons of the forms automatically set values for some of the form's hidden
fields using some javascript code connected to the \emph{onClick} action.
To overcome this issue, a new ad-hoc submit button is created inside the form
and it is used to perform the action on behalf of the original one. This functionality
is implemented inside the \lword{\emph{src/common/utils.java}} file with the \emph{addSubmitButton}
function.


\section{Source code fixes}\label{fixes}
All the vulnerabilities explained in Section~\ref{tests} are exploited in the test cases
inserting a malicious link that appears on the page. This succeeds because
the application does not perform any kind of validation and/or sanitization on the variables
that are shown on the page. Unfortunately, many of these variables are (or contain) input
under the control of the user. Thus, performing sanitization steps is not an option in this
case.

The sanitization phase can be performed either before the user input is stored into the
MySQL database or when the value is retrieved from the storage to be displayed on the
web page. This is up to the developer and highly depends on the application structure.
To fix the PHP source code of \emph{Schoolmate}, this work makes use of the first approach.

The \emph{index.php} file is the one responsible for handling every request coming from
every other page in the application and it is the one including all the other ones depending
on the values of $page$ and $page2$.
This architectural decision can be exploited in order
to fix all the vulnerabilities affecting the entire application in a single point.
The portion code shown below performs the sanitization step
of all the POST parameters reaching the application.

\begin{lstlisting}[frame=single, caption={Fixing all the vulnerabilities in index.php}]
<?php
foreach ($_POST as $k => $val) {
 $_POST[$k] =
    htmlentities($val,ENT_QUOTES,"UTF-8");
}
...
?>
\end{lstlisting}

The decision to apply the \textbf{htmlentities}\footnote{The \emph{htmlentities} function is preferred
to the \emph{htmlspecialchars} since the first one is a ``stronger'' version of the second. More specifically,
the \emph{htmlentities} encodes every character passed as argument and not only the ones that have a special
meaning in HTML.} function to all the items in the POST array allows
to eliminate the possibility to modify the page's content by means of proper malicious tags in the input
text fields that are later printed on the page. This countermeasure prevents the attaker from being able to
alter the DOM structure of the web page. This function converts all the characters, including the ones
that has a special meaning in HTML, to its ``safe'' encoded version. The optional parameter
\emph{ENT\_QUOTES} is used to convert both double and single quotes, while the \emph{UTF-8}
specities the encoding to be used when converting the characters.\\
Finally, the addition of this code snippet at the very beginning of the $index.php$ file allows to
fix both all the stored and reflected XSS vulnerabilities in $Schoolmate$, since these were the
only vulnerabilities of interest in this security revision.


\section{Conclusions}\label{outcomes}
The test cases developed for this work and explained in Section~\ref{tests} exploits all the
vulnerabilities found by Pixy which are classified as true positives. All the tests failed when
the original code of $Schoolmate$ is deployed on the webserver, while they all successfully pass
once the fixes explained in Section~\ref{fixes} are applied.

In order to check that the patch does not introduce any new vulnerability and fixes the old ones,
the static analysis tool is run again on the fixed version of $Schoolmate$.\\
After this final step, Pixy revealed the same vulnerabilities discussed in Section~\ref{tests}.
Despite the outcome of the tool, all the PoC tests successfully pass after patching the application.
Hence, it is possible to claim that all the reports created by Pixy are false positives.
This is due to the fact that static anaysis
tools perform an overestimation of the paths that can be followed during an actual execution
of the application\footnote{As explained in Section~\ref{tests}, this is the reason why an initial
dinstinction between TP and FP has to be done in advance before implementing the PoC attacks.}.\\
Thus, based on the outcome of the test suite, it is possible to state that the vulnerabilities are
fixed and that they do not exist anymore.


\section{Preliminary steps}\label{steps}
In order to run all the test cases, the following preliminary steps has to be followed
to fullfil the application's requirements:
\begin{enumerate}
    \item install and run a webserver (eg. Apache), PHP and MySQL.
    \item edit the $\$dbuser$, $\$dbpass$ and $\$dbaddress$ variables inside the $index.php$ file in
        order to match the parameters of the MySQL instance to be used. The right $\$dbname$ is set
        by the SQL script described in step number \ref{sqlscript}.
    \item copy the Schoolmate source files inside the webserver directory.
    \item \label{sqlscript} import the $dbDump.sql$ script into the MySQL instance. It takes care of
        creating the a database with the right name and populating it with some initial values.
        All the queries on the database should succeed.
    \item import the tests' source code as a Maven project in your IDE so that all the
        needed dependencies can be automatically downloaded from the remote repository
        and included in the project.
\end{enumerate}
After all the abovementioned steps are completed successfully, Schoolmate is up and running
and the test suite can be run.


%\end{document}  % This is where a 'short' article might terminate
%
% The following two commands are all you need in the
% initial runs of your .tex file to
% produce the bibliography for the citations in your paper.

%\bibliographystyle{abbrv}
%\bibliography{biblio}% biblio.bib is the name of the Bibliography in this case

% You must have a proper ".bib" file
%  and remember to run:
% latex bibtex latex latex
% to resolve all references
%
% ACM needs 'a single self-contained file'!
%
\end{document}

