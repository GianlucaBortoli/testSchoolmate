\documentclass{sig-alternate-05-2015}
\usepackage{subscript}
\usepackage{tikz}
\usepackage{hyperref}
\usepackage{graphicx}
\usepackage{amsmath}
\usepackage{listings}
\usepackage[lined,boxed,longend,ruled]{algorithm2e}
\newcommand\mycommfont[1]{\footnotesize\ttfamily\textcolor{blue}{#1}}
\SetCommentSty{mycommfont}
\SetKwProg{Fn}{def}{\string:}{end}
\SetKwFunction{CompGrid}{compute\_grid}
\SetKwFunction{CompTop}{compute\_topics}
\SetKwFunction{CompComArea}{compute\_intersection}
\SetKwFunction{CompDiffArea}{compute\_difference}
\SetKwFunction{CrDT}{create\_dataframes}
\SetKwFunction{SeqCom}{sequential\_computation}
\SetStartEndCondition{ }{}{}%
\SetKw{KwTo}{in}\SetKwFor{For}{for}{\string:}{end for}%
\newcommand{\forcond}{$df$ \KwTo{$dist\_clust\_pages$}}
\newcommand{\forRecDiff}{$map$ \KwTo{$commonAreas$}}
\newcommand{\forRecCom}{$map$ \KwTo{$computedMaps$}}
\newcommand{\forNaive}{$page$ \KwTo{$pages$}}
\newcommand{\forNaiveCom}{$cell$ \KwTo{$squares$}}
\newcommand\lword[1]{\leavevmode\nobreak\hskip0pt plus\linewidth\penalty50\hskip0pt plus-\linewidth\nobreak#1}
\AlgoDontDisplayBlockMarkers\SetAlgoNoEnd\SetAlgoNoLine%
\graphicspath{ {./imgs/} }

% listings PHP code color
\definecolor{dkgreen}{rgb}{0,.6,0}
\definecolor{dkblue}{rgb}{0,0,.6}
\definecolor{dkyellow}{cmyk}{0,0,.8,.3}
\lstset{
  language        = php,
  basicstyle      = \small\ttfamily,
  keywordstyle    = \color{dkblue},
  stringstyle     = \color{red},
  identifierstyle = \color{dkgreen},
  commentstyle    = \color{gray},
  emph            =[1]{php},
  emphstyle       =[1]\color{black},
  emph            =[2]{if,and,or,else},
  emphstyle       =[2]\color{dkyellow}
}


\begin{document}

\title{Security testing report}

\numberofauthors{1}
\author{
    Gianluca Bortoli\\
           \affaddr{DISI\,-\,University of Trento}\\
           \affaddr{Student id: 179816}\\
           \email{gianluca.bortoli@studenti.unitn.it}
}
\maketitle


\section{Introduction}
This work aims at performing a security analysis study on \emph{Schoolmate}.
This web service is PHP/MySQL solution for elementary,
middle and high schools where different type of users can manage all the needed
information to fulfill their job.

This report is structured as follows. Section \ref{tests} describes the naming
convention used for the test cases and how they are structured into packages.
Section \ref{fixes} describes the vulnerabilities that have been found on the
application, their root causes and how to fix them. Section \ref{outcomes} reports
the results of the proof-of-concept (PoC) attacks described in Section \ref{tests}.
Finally, Section \ref{steps} depicts the steps that have to be taken in order to
be able to run all the tests from scratch, while Section \ref{conclusions} gives
a general evaluation of the whole work.


\section{Security test cases}\label{tests}
Pixy is a scanner static code analysis tools that scans PHP applications for
security vulnerabilities. This software is used in order to spot the possible
vulnerabilities in \emph{Schoolmate}, the subject web application.

The workflow followed developing this work is the following:
\begin{enumerate}
    \item run Pixy on the application code to produce vulnerability reports
        regarding Cross Site Scripting attacks (XSS).
    \item \label{classification} manually analyze all the reports to classify the vulnerabilities.
        Thus, each of them has to be categorized as \textbf{false positive},
        \textbf{reflected XSS} or \textbf{stored XSS} (see \emph{xss\_classification.pdf
        attachment}).
    \item write all the test cases for all the true positives (TP) found (ie. the
        vulnerabilities classified either as reflected or stored XSS) in the form
        of a PoC attack against the application under test.
    \item check that every test case fails on the bugged application to demonstrate
        \emph{Schoolmate} has such security vulnerabilities.
    \item \label{fix} modify the \emph{Schoolmate}'s PHP code to fix the vulnerabilities found
        in step \ref{classification}.
    \item run again the test suite and check all the tests pass, meaning that
        the PoC attack implemented in each test case is no longer possible.
    \item run again Pixy on the fixed application source code to check the
        vulnerabilities are really fixed and making sure step \ref{fix} did not
        introduce any security breach.
\end{enumerate}
The test suite developed to demonstrate that the application under revision has
security concerns makes use of JWebUnit, a Java-based testing framework for web
applications. Every Pixy report is identified by a number and may contain multiple
vulnerabilities. The test suite's source code is structured as follows:
\begin{itemize}
    \item the \textbf{src/main/java} folder contains a Java file for each Pixy report which
        has at least one TP. The test-case Java file naming follows the convention
        \lword{\emph{Test<PixyReportNumber>.java}} so that it is possible to easily match the test
        case with its Pixy report counterpart.
    \item the \textbf{src/main/java/common} folder contains three packages that implement
        utility functions that are available to all the test cases. These functions
        allows not to soil the actual test case implementation, separating all the
        sequences of operations that has to be done in many tests in order to navigate
        the web page depending on the situation.
\end{itemize}
A noteworthy aspect is how tests' assertions are written. When dealing with
security PoC test cases, the security expert writes a test that is thought to fail
if the vulnerability can be exploited and pass otherwise. Hence, in this scenario
a failing test (and its assertion) means that the vulnerability is still present,
while a passing one means that the vulnerability is not present any more.


\section{Source code fixes}\label{fixes}
\begin{lstlisting}[frame=single, caption={Fixing all the vulnerabilities in index.php}]
<?php
foreach ($_POST as $k => $val) {
 $_POST[$k] =
    htmlentities($val, ENT_QUOTES, "UTF-8");
}
...
?>
\end{lstlisting}


\section{Testing outcomes}\label{outcomes}

\section{Preliminary steps}\label{steps}

\section{Conclusions}\label{conclusions}


%\end{document}  % This is where a 'short' article might terminate
%
% The following two commands are all you need in the
% initial runs of your .tex file to
% produce the bibliography for the citations in your paper.

%\bibliographystyle{abbrv}
%\bibliography{biblio}% biblio.bib is the name of the Bibliography in this case

% You must have a proper ".bib" file
%  and remember to run:
% latex bibtex latex latex
% to resolve all references
%
% ACM needs 'a single self-contained file'!
%

\end{document}
